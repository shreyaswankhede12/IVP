\section{Literature Review}
\label{sec:LiteratureReview}

Old popular data hiding techniques have generally revolved around modifying the least significant bit (LSB) of the pixel values of an image, which often overwrites visual structures and makes the hidden messages vulnerable to visual detection methods. The importance of Fourier Transformation in data hiding techniques has been seen rarely in the literature. In early work, researchers have used it by encoding the hidden message in coefficients of the Discrete Fourier Transform (DFT) magnitude. This approach overcomes the shortcomings of old techniques by exploiting the spectral properties of Fourier Transformation, improving capacity, security, and robustness.

An important fact that the paper builds on is that as long as the Fourier phase of an image remains intact, the original appearance remains intact with very minute visible artifacts if the Fourier magnitude of the image is slightly modified. Experimental results show that the hidden message image may be as large as half the size of the carrier image for an unnoticeable amount of noise artifacts in the stego image.

Experimental evidence has established that the human visual system is much more sensitive to brightness than color. Therefore, the color and brightness separation strategy is used, avoiding altering the luminance channel. Altering the chrominance channels of an image has very few visible artifacts, almost not detectable by the human eye. Thus, the hidden message is embedded into the Fourier magnitude spectrum of the chrominance channels of the image.

The paper presents a method of transforming the image into the Lab* space non-linearly, following the color/brightness separation strategy. This transformation specifies the color of an image into human perception independently of any particular imaging device. The non-linear operation converts the RGB tuple into the Lab tuple, where L is the luminance channel, and a and b are the chrominance channels. This technique has almost double the hidden data carrying capacity than previous techniques. The DFT of the chrominance channels a and b is taken to embed the secret data, separating magnitude and phases in each of the chromatic channels. The hidden message is embedded by replacing high-frequency areas of chrominance channels in the Fourier magnitude of the chrominance channels, preventing aliasing when extracted. After modifying the Fourier magnitude of a and b chrominance channels and combining them with their corresponding phases, the Inverse Discrete Fourier Transform (IDFT) is applied to get the modified color/brightness-separated image tuple (L ,a' ,b'). Then, this tuple is transformed into the modified image tuple (R',G',B'), resulting in the stego image S = (R',G',B').

To recover the hidden information, the process is reversed. First, the stego image is converted into the Lab* space by the non-linear transformation used during the hiding phase. Then, the DFT of the chromatic channels a' and b' is taken, and the security key for-loop is applied to recover the hidden message. The first part of the hidden message is extracted from the high-frequency areas in the Fourier chrominance-a magnitude spectrum, and the second hidden image is extracted from the high-frequency areas in the Fourier chrominance-b magnitude spectrum. The technique provides a 3-layered security measure because the resultant image scatters the hidden image three times across all pixels of the carrier image. This scattering leads to robustness to stego medium tampering. As long as 40\% or more of the stego image data remains intact, the hidden message image can be extracted with reasonable integrity. The paper demonstrates the robustness of this image hiding technique to various tampering effects, such as cutting parts of the image, repainting the image, and rotating the stego image, while still being able to recover the message.

In general, a technique qualifies as a steganography technique broadly based on:
\begin{enumerate}
  \item Transparency: How much information you can hide in the cover image (stego hidden) without distorting the original appearance of the image much so that it is undetectable by any visual attacks.
  \item Robustness/Tamper resistance: The ability of the hidden message to remain undamaged even if the stego image undergoes some sort of transformation like filtering (linear and blurring), blurring, cropping, repainting.
\end{enumerate}
We can see clearly that the technique mentioned in the paper satisfies the above properties. In total, the paper presents a very secure, high-capacity, tamper-resistant technique to hide some secrecy exploiting the color/brightness of the image and spectral properties of the Fourier-transformed image.
