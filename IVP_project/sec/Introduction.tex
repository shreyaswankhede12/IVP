\section{Introduction}
\label{sec:intro}
In today's digital age, the importance of secure communication and data transmission cannot be overstated. With the proliferation of digital platforms and the vast exchange of information over networks, ensuring the confidentiality and integrity of sensitive data has become paramount. This has led to the emergence of various techniques and methodologies aimed at concealing information within digital media, one of which is steganography.

Steganography, derived from the Greek words "steganos" (meaning covered or concealed) and "graphein" (meaning writing or drawing), is the art and science of hiding secret information within innocuous cover media in such a way that the existence of the hidden message remains undetectable to unintended recipients. Unlike cryptography, which focuses on encrypting the content of a message to render it unintelligible to unauthorized parties, steganography ensures that the existence of the hidden information itself is not apparent.

Steganography techniques can be broadly categorized into various types based on the cover media, such as text, audio, video, and images. In this project, we specifically focus on digital image steganography, leveraging the frequency domain approach facilitated by techniques like Discrete Fourier Transform (DFT).

Digital image steganography involves embedding secret data into digital images, exploiting the imperceptible alterations that can be made to the pixel values or the frequency components of the image. By utilizing transformations like DFT, which represent images in terms of their frequency components, we can embed information in a manner that is robust against typical image processing operations while minimizing perceptual distortion.

By employing the frequency domain approach, we aim to achieve efficient and imperceptible hiding of information within images, thereby ensuring both the secrecy and integrity of the concealed data.

